\documentclass[10pt]{article}
\usepackage{comment,amsmath}
\usepackage{graphicx,parskip}
\usepackage{enumitem,todonotes}
\usepackage{sometexdefs}

\providecommand{\tightlist}{%
  \setlength{\itemsep}{0pt}\setlength{\parskip}{0pt}}

%\usepackage[spanish]{babel}
\usepackage{fontspec}
\setmainfont{Fira Sans}
\usepackage{microtype}

%\bibliographystyle{apalike}
\usepackage[letterpaper,bottom=.75in,top=1in,right=1in,lmargin=1.5in]{geometry}

\usepackage[style=authoryear, backend=biber]{biblatex}
\addbibresource{classbib.bib}

\usepackage[mmddyyyy]{datetime}
\usepackage{advdate}

\usepackage{soul}
\usepackage[compact,nobottomtitles*]{titlesec} %nobottomtitles
\titleformat{\part}[hang]{\bfseries\large\scshape}{\hspace{-.5in}\thepart}{.5em}{}{}
\titleformat{\section}[hang]{\large\bfseries}{\hspace{-.5in}\thesection ---}{.25em}{}{}
%\titleformat{\subsection}[wrap]{\small\bfseries}{\thesubsection}{.5em}{}{}
\titleformat{\subsubsection}[leftmargin]{\itshape\filleft}{\thesubsubsection}{.2em}{\hspace{-.75in}}{}
\titleformat{\paragraph}[runin]{\bfseries}{\theparagraph}{0em}{}{}

%\titlespacing{\part}{0ex}{.5ex plus .1ex minus .2ex}{.25\parskip}
%\titlespacing{\section}{0ex}{1.5ex plus .1ex minus .2ex}{.25\parskip}
\titlespacing*{\section}{-.5in}{1em}{0em}{}%
%\titlespacing{\subsection}{0ex}{.5ex plus .1ex minus .1ex}{1ex}
%\titlespacing{\subsection}{2pc}{1.5ex plus .1ex minus .2ex}{1pc}
%\titlespacing{\subsection}{12pc}{1.5ex plus .1ex minus .2ex}{1pc}
\titlespacing{\subsubsection}{0ex}{.5ex plus .1ex minus .1ex}{1ex}
\titlespacing{\paragraph}{0em}{1ex}{.5ex plus .1ex minus .1ex}

\newcommand{\secformat}[1]{\MakeLowercase{\so{#1}}}
   % \so spaces out letters
%\titleformat{\subsection}[block]
%  {\normalfont\scshape\filcenter}
%  {\thesection}
%  {1em}
%  {\secformat}

  \titleformat{\subsection}[leftmargin]
  {\small
   \vspace{6pt}%
   \sffamily\bfseries\filleft}
  {\thesection}{.5em}{}

\titlespacing{\subsection}{4pc}{1.5ex plus .1ex minus .2ex}{1pc}

\newenvironment{introstuff} {\setcounter{secnumdepth}{0}} {\setcounter{secnumdepth}{1}}

% Create new title appearance
\makeatletter
\def\maketitle{%
    %\null
    \thispagestyle{empty}%
    \begin{center}\leavevmode
        \normalfont
        {\large \bfseries\@title\par}%
        {\large \@author\par}%
        {\large \@date\par}%
    \end{center}%
\null }
\makeatother

\usepackage{fancyhdr}
% \renewcommand{\sectionmark}[1]{\markright{#1}{}}

\fancypagestyle{myfancy}{%
    \fancyhf{}
    % \fancyhead[R]{\small{Page~\thepage}}
    \fancyhead[R]{\small{Statistical Inference -- Spring 2023-- \thepage}}
    \fancyfoot[R]{\footnotesize{Version~of~\input{|"date"}}}
    % \fancyfoot[R]{\small{\today -- Jake Bowers}}
    \renewcommand{\headrulewidth}{0pt}
\renewcommand{\footrulewidth}{0pt}}


\newcommand{\entrylabel}[1]{\mbox{\textsf{#1:}}\hfil}

%% These next lines tell latex that it is ok to have a single graphic
%% taking up most of a page, and they also decrease the space arou
%% figures and tables.
\renewcommand\floatpagefraction{.9} \renewcommand\topfraction{.9}
\renewcommand\bottomfraction{.9} \renewcommand\textfraction{.1}
\setcounter{totalnumber}{50} \setcounter{topnumber}{50}
\setcounter{bottomnumber}{50} \setlength{\intextsep}{2ex}
\setlength{\floatsep}{2ex} \setlength{\textfloatsep}{2ex}

\specialcomment{com} {\begingroup\sffamily\small\bfseries}{\endgroup}
\excludecomment{com}

\title{Introduction to the Design and Analysis of Randomized Experiments}

\author{Instructor: Jake Bowers \\
    jwbowers@illinois.edu \\
    \url{http://jakebowers.org/}
}

\date{Escuela de Invierno de Métodos \the\year}

\usepackage[xetex,colorlinks=TRUE,citecolor=blue]{hyperref}

%\renewcommand{\bibname}{ }
% \renewcommand{\refname}{\normalsize{Required:}}
%\renewcommand{\refname}{\vspace{-2em}}

\def\themonth{\ifcase\month\or
        January\or February\or March\or April\or May\or June\or
    July\or August\or September\or October\or November\or December\fi}

    \begin{document}
    \pagestyle{myfancy}
    %\newgeometry{lmargin=1.5in}     % use whatever margins you want for left, right, top and bottom.

 \begin{introstuff}

        \maketitle

        \part*{Overview}

This course introduces the basic statistical concepts that guide the design and
analysis of randomized experiments. The act of randomizing the assignment of an
intervention offers special benefits to researchers interested in making
counterfactual causal inferences, and the course begins by engaging with
questions about why randomize (or why not randomize) and how randomized
assignment is not the same as random sampling. It then introduces randomization
based statistical inference; an approach to calculating standard errors for
average treatment effect estimates and to calculating $p$-values for tests of
hypotheses about causal effects. Discusses power analysis, and ends by engaging
with some of the trickier issues in experimental design and analysis: what to
do when experimental units drop out of the study or otherwise do not provide
valid outcomes? how might we think about making causal inferences when the
active treatment, the treatment of theoretical interest, cannot be directly
randomized? what to do when we cannot randomly assign treatment directly to
individual units but only to groups of them? Throughout we will be using the R
programming language to demonstrate statistical concepts and also as tools for
designing and analyzing randomized experiments. Most of the examples in the
class will come from field experiments but given the focus on fundamentals the
class will be useful for those working with randomized survey experiments and
randomized lab experiments as well.

I will be open to adding and subtracting topics depending on student interest.

\section{Goals and Expectations}

This class aims to help you get started with the design and analysis of
randomized experiments using randomization as the basis for statistical
inference.

The point of the course is to position you to do the future learning
that is at the core of your work as a researcher. That is,
for most of your life you will not have classes or even textbooks
available to help you learn how to work with data and statistics.
Rather, you will have to learn on your own, with the help of the
internet, friends, and friendly AI. So, this course aims to help you
learn how to learn even as it helps you learn how to reason, decide and
evaluate.

The \textbf{specific goals} of the course are that students:

\begin{itemize}[noitemsep]

    \item Explain in their own words key concepts in statistics like "causal
        inference","statistical inference", "hypothesis testing", "point
        estimation", "p-value", "confidence interval", "random assignment", and
        describe how such concepts fit together in applied research.
    \item Have practiced coding randomization, power calculations, and statistical analysis
        of experimental results
    \item Be familiar with standards and practices for many aspects of reproducible research
    \item Have practiced developing an experimental research design
    \item Have practiced created a pre-analysis plan for an experimental design

\end{itemize}


\subsection{Books}

I will recommend that you read chapters from the following books:

\begin{verse}
\fullcite{gerber2012field}

\fullcite{bowersVoorsIchino2021book}  \href{https://egap.github.io/theory_and_practice_of_field_experiments_spanish/}{En español}

\fullcite{blair2023research}
\end{verse}

I also recommend some chapters from
\href{https://www-jstor-org.eui.idm.oclc.org/stable/j.ctt4cgd52}{\cite{glennerster13}},
which presents material similar to Gerber and Green but less technically.
Reading this volume alongside Gerber and Green might bring together intuition
and statistics for some students.

Other readings will be assigned and distributed electronically.

\subsection{Assumptions}

        I assume some previous engagement with high school mathematics,
        probability and statistical computing in the R statistical programming
        language. If you haven't had experience with R but you love learning
        computing languages then you can still get a lot out of this course ---
        you will learn a lot about R as kind of laboratory for learning about
        statistical theory and evaluating and analyzing experimental designs.


If you're new to R and/or statistics, I suggest using books like these or other resources for learning about R and basic statistics online.

\begin{verse}
    \fullcite{fox2016}

    \fullcite{wickhamgrolemund2017}
\end{verse}


\subsection{Credits}

Those who want to take this course for university credit should plan to create
a proposal for an experimental design and/or a pre-analysis plan. I recommend
meeting with me at least once during the week or even before the week of the
course to discuss this.  I will be presenting a template for research design
during the class sessions.

\subsection{Expectations}

\textbf{Ask questions when you don't understand things; chances are you're not
alone.}

 \end{introstuff}
    %\restoregeometry

 \part*{Schedule}

        \textbf{Note: } This schedule is preliminary and subject to change. If
        you miss a class make sure you contact me or one of your colleagues to
        find out about changes in the lesson plans or assignments.

\titleformat{\subsection}[hang]{\sffamily\bfseries}{\thesubsection}{}{}{}
\titleformat{\subsubsection}[hang]{\small\sffamily\bfseries}{\thesubsubsection}{}{}{}
\titlespacing{\subsection}{0pt}{1.5ex plus .1ex minus .2ex}{0pt}

\section{Causal inference and the potential outcomes framework}
% Maybe Google form before class with favorite randomized experiment.

What is causality, the potential outcomes framework, and how do experiments help causal identification?
\subsection{Useful readings}

\begin{itemize}
    \tightlist
    \item \fullcite{gerber2012field}, Ch 1--3
    \item \textbf{Application:} A lab experiment: \href{https://journals.sagepub.com/doi/pdf/10.1177/0956797613495244}{\fullcite{aaroe_hunger_2013}} which uses a randomized experiment to follow the nicely designed observational study in  \href{https://doi.org/10.1111/pops.12062}{\fullcite{petersen_social_2014}}

    \item \textbf{Application:} A field experiment: \fullcite{gerber2008social}
    \item \textbf{Application:} An audit-experiment: \fullcite{bertrand_are_2004}

    \item \textbf{Recommended:}
        \href{https://egap.github.io/theory_and_practice_of_field_experiments/}{\fullcite{bowersVoorsIchino2021book}},
        Module 1--3 
    \item \textbf{Recommended:} \fullcite{glennerster13}, chs. 1--3
    \item \textbf{Recommended:} \fullcite{banerjee2011poor}, ch. 1
    \item \textbf{Recommended:} \fullcite{rosenbaum2020book} Sections 2.1--2.4
    \item \textbf{Recommended:} \fullcite[Chap 2--3]{rosenbaum2017}
    \item \textbf{Recommended:} \fullcite[Chap 2--2.4]{rosenbaum2002book} explains and formalizes Fisher's randomization inference.
    \item \textbf{Recommended:} \fullcite[Chap 2]{fisher:1935} explains \emph{the} invention of
    random-assignment based randomization inference in about 15 pages.
    \item \textbf{Recommended:} \fullcite{neyman:1923,rubin1990apt} \emph{the} invention of random-sampling
    based randomization inference (estimators of the average causal effect and
    their standard errors).
    \item \textbf{Recommended:} \fullcite{bowersleavitt2020} Provides some statistical theory that connects Fisher's test and Neyman's
     estimator for causal inference.
\end{itemize}

\section{Random assignment and identification under randomization}

How does randomization help us learn about causal effects? Randomization
strategies (simple, complete, blocked, clustered, etc.), random sampling versus
random assignment.

\subsection{Useful readings}

\begin{itemize}
    \item \fullcite{gerber12}, ch. 2, remaining sections
    \item \fullcite{karlan16} ch. 2 (skim).
    \item \textbf{Recommended:}  \fullcite{glennerster13} ch. 4.
    \item \textbf{Recommended:} \fullcite{wantchekon2003clientelism}
    \item \textbf{Recommended:} \fullcite{collier2014votes}
        \item \textbf{Recommended:}
            \fullcite{bowersVoorsIchino2021book}, Module 4 materials on randomization.
\end{itemize}

\section{Statistical inference for counterfactual causal effects}

More on the statistics of randomized experiments focusing on statistical power
but reviewing testing and estimation. Sampling distributions, causal inference,
hypothesis testing and statistical power.

\subsection{Useful readings}

\begin{itemize}
    \item \fullcite{gerber12}, ch. 3.
    \item \textbf{Recommended:}
        \fullcite{bowersVoorsIchino2021book}, Module 5 \& 6 materials on hypothesis testing and estimation.
    \item \cite{karlan16} ch. 5.
    \item \textbf{Recommended:}  \fullcite{glennerster13} ch. 6.
    \item \textbf{Recommended:}
        \href{https://egap.github.io/theory_and_practice_of_field_experiments/}{\cite{bowersVoorsIchino2021book}},
        Module 7 materials.
\end{itemize}


\section{Clustered and blocked designs and power}

How to think about designs where the researchers cannot directly administer a
treatment to an individual unit of interest (cluster randomization)?  And/or
why we might construct an overall experiment from a set of mini-experiments
(blocking or stratification)?

% Natalia already discussed these designs and randomization for them. Use
% blocking and clustering to discuss power.

% Main lecture: Nahomi and Maarten
% Review why blocking, why cluster-randomization, costs and benefits
% Rwanda One example

% Small groups (30 min):
% apply concepts to the 2nd paper (Maarten paper)
% How are the papers dealing with blocks and
% clusters? Core assumptions from Day 1 to flag?
% Small groups link papers to concepts ("How many blocks? How many clusters? Why
% cluster? Why block?"

\subsection{Useful readings}

\begin{itemize}
    \tightlist
    \item \fullcite{gerber12}, ch.8, Sections 8.1--8.3 only
    \item \href{https://www.journals.uchicago.edu/doi/pdfplus/10.1017/S0022381611001368}{\fullcite{ichino2012deterring}}
        % Blocking not clustering
        % Careful attention to the estimands
    \item \href{https://doi.org/10.1093/qje/qjaa039}{\fullcite{christensen_building_2021}}
        % Multiple levels, polygons and randomization
        % Mahalnobis distance triple (cite Moore)
   % \item \textbf{Recommended:} Pre-analysis plan for Christensen et al.~(2021)
\end{itemize}

% Recommended:

%\item \fullcite{paluck_deference_2009}
% Notes on paluck etc lots of kinds of measurement, behavioral outcome
% 14 clusters, 40 people per cluster, 7 pairs.
% Different test statistics


\section{Maybe: Survey experiments}

We could spend a session focusing on some of the challenges and opportunities
posed by randomized experiments embedded within surveys.

% Main lecture: Jake (Maarten and Nahomi assigned to do tasks/parts)

% Issues of Interpretation
% complex noncompliance (mediation, defiers, etc..)
% spillovers
% attrition
% factorial designs
% survey experiments

% Allan Dafoe, Baobao Zhang, and Devin Caughey. “Information equivalence in survey experiments”. In: Political Analysis 26.4 (2018), pp. 399–416

\subsection{Useful readings}

% Readings
\begin{itemize}
    \tightlist
    \item \href{https://gustavodiaz.org/files/research/sage_survey_experiments_chapter.pdf}{\fullcite{diazsurvey2020}}
    \item \href{https://egap.org/resource/10-things-to-know-about-survey-experiments/}{\fullcite{grady2019survey}}
    \item \fullcite{teele_ties_2018}
    \item \textbf{Recommended:} \fullcite{gaines2007logic}
    \item \textbf{Recommended:} \fullcite{winters2013lacking} and \fullcite{boas2019norms} and \fullcite{incerti2020corruption}.
\end{itemize}


% Exercises

    \section{Maybe: Causal effects of non-random compliance with randomized interventions}

    If an intervention has been randomized but the active
    dose has not, what can we do? It turns out that in this case we can
    estimate Complier Average Causal Effects (CACE) (also known as the Local
    Average Treatment Effect (LATE)) and we can also test hypotheses about the
    causal effects of non-randomly taking a dose of a randomized treatment.

Instrumental variables and the placebo controlled design and experiments built
on previously executed randomizations (downstream experiments).

\subsection{Useful readings}
\begin{itemize}
    \tightlist
    \item \fullcite{gerber12}, ch. 5
    \item \href{https://doi.org/10.1111/ecca.12168}{\fullcite{friedman_education_2016}} (Instrumental variables) % try to find something from Latin America?
    \item \href{https://www-cambridge-org.eui.idm.oclc.org/core/services/aop-cambridge-core/content/view/S0003055419000923}{\fullcite{kalla_reducing_2020}} (A placebo, control, treatment)
    \item Chapters 5 and 6 of \fullcite{gerbergreen2012}.
    \item Chapter 5 and especially Section 5.3, ``Instruments,'' of \fullcite{rosenbaum2020book}
    \item Chapter 13, ``Instruments'' of \fullcite{rosenbaum2017}
    \item \fullcite{sovey2011instrumental}
    \item \fullcite{rosenbaum1996}
    \item \fullcite{angristetal1996}
    \item \fullcite{imbensrosenbaum2005} on weak instruments and the problem of 2SLS as an estimator
    \item \fullcite{kangpeckkeele2018}
\end{itemize}

\section{Maybe: Student presentations of experimental research designs and/or pre-analysis plans}
\newpage

\printbibliography[title=References]

\end{document}

