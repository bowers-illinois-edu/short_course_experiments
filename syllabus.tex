\documentclass[10pt]{article}
\usepackage{comment,amsmath}
\usepackage{graphicx,parskip}
\usepackage{enumitem,todonotes}
\usepackage{sometexdefs}

%\usepackage[spanish]{babel}
\usepackage{fontspec}
\setmainfont{Fira Sans}
\usepackage{microtype}

%\bibliographystyle{apalike}
\usepackage[letterpaper,bottom=.75in,top=1in,right=1in,lmargin=1.5in]{geometry}

\usepackage[style=authoryear, backend=biber]{biblatex}
\addbibresource{classbib.bib}

\usepackage[mmddyyyy]{datetime}
\usepackage{advdate}

\usepackage{soul}
\usepackage[compact,nobottomtitles*]{titlesec} %nobottomtitles
\titleformat{\part}[hang]{\bfseries\large\scshape}{\hspace{-.5in}\thepart}{.5em}{}{}
\titleformat{\section}[hang]{\large\bfseries}{\hspace{-.5in}\thesection ---}{.25em}{}{}
%\titleformat{\subsection}[wrap]{\small\bfseries}{\thesubsection}{.5em}{}{}
\titleformat{\subsubsection}[leftmargin]{\itshape\filleft}{\thesubsubsection}{.2em}{\hspace{-.75in}}{}
\titleformat{\paragraph}[runin]{\bfseries}{\theparagraph}{0em}{}{}

%\titlespacing{\part}{0ex}{.5ex plus .1ex minus .2ex}{.25\parskip}
%\titlespacing{\section}{0ex}{1.5ex plus .1ex minus .2ex}{.25\parskip}
\titlespacing*{\section}{-.5in}{1em}{0em}{}%
%\titlespacing{\subsection}{0ex}{.5ex plus .1ex minus .1ex}{1ex}
%\titlespacing{\subsection}{2pc}{1.5ex plus .1ex minus .2ex}{1pc}
%\titlespacing{\subsection}{12pc}{1.5ex plus .1ex minus .2ex}{1pc}
\titlespacing{\subsubsection}{0ex}{.5ex plus .1ex minus .1ex}{1ex}
\titlespacing{\paragraph}{0em}{1ex}{.5ex plus .1ex minus .1ex}

\newcommand{\secformat}[1]{\MakeLowercase{\so{#1}}}
   % \so spaces out letters
%\titleformat{\subsection}[block]
%  {\normalfont\scshape\filcenter}
%  {\thesection}
%  {1em}
%  {\secformat}

  \titleformat{\subsection}[leftmargin]
  {\small
   \vspace{6pt}%
   \sffamily\bfseries\filleft}
  {\thesection}{.5em}{}

\titlespacing{\subsection}{4pc}{1.5ex plus .1ex minus .2ex}{1pc}

\newenvironment{introstuff} {\setcounter{secnumdepth}{0}} {\setcounter{secnumdepth}{1}}

% Create new title appearance
\makeatletter
\def\maketitle{%
    %\null
    \thispagestyle{empty}%
    \begin{center}\leavevmode
        \normalfont
        {\large \bfseries\@title\par}%
        {\large \@author\par}%
        {\large \@date\par}%
    \end{center}%
\null }
\makeatother

\usepackage{fancyhdr}
% \renewcommand{\sectionmark}[1]{\markright{#1}{}}

\fancypagestyle{myfancy}{%
    \fancyhf{}
    % \fancyhead[R]{\small{Page~\thepage}}
    \fancyhead[R]{\small{Statistical Inference -- Spring 2023-- \thepage}}
    \fancyfoot[R]{\footnotesize{Version~of~\input{|"date"}}}
    % \fancyfoot[R]{\small{\today -- Jake Bowers}}
    \renewcommand{\headrulewidth}{0pt}
\renewcommand{\footrulewidth}{0pt}}


\newcommand{\entrylabel}[1]{\mbox{\textsf{#1:}}\hfil}

%% These next lines tell latex that it is ok to have a single graphic
%% taking up most of a page, and they also decrease the space arou
%% figures and tables.
\renewcommand\floatpagefraction{.9} \renewcommand\topfraction{.9}
\renewcommand\bottomfraction{.9} \renewcommand\textfraction{.1}
\setcounter{totalnumber}{50} \setcounter{topnumber}{50}
\setcounter{bottomnumber}{50} \setlength{\intextsep}{2ex}
\setlength{\floatsep}{2ex} \setlength{\textfloatsep}{2ex}

\specialcomment{com} {\begingroup\sffamily\small\bfseries}{\endgroup}
\excludecomment{com}

\title{Introduction to the Design and Analysis of Randomized Experiments}

\author{Instructor: Jake Bowers \\
    jwbowers@illinois.edu \\
    \url{http://jakebowers.org/}
}

\date{Escuela de Invierno de Métodos \the\year}

\usepackage[xetex,colorlinks=TRUE,citecolor=blue]{hyperref}

%\renewcommand{\bibname}{ }
% \renewcommand{\refname}{\normalsize{Required:}}
%\renewcommand{\refname}{\vspace{-2em}}

\def\themonth{\ifcase\month\or
        January\or February\or March\or April\or May\or June\or
    July\or August\or September\or October\or November\or December\fi}

    \begin{document}
    \pagestyle{myfancy}
    %\newgeometry{lmargin=1.5in}     % use whatever margins you want for left, right, top and bottom.

 \begin{introstuff}

        \maketitle

        \part*{Overview}

This course introduces the basic statistical concepts that guide the design and
analysis of randomized experiments. The act of randomizing the assignment of an
intervention offers special benefits to researchers interested in making
counterfactual causal inferences, and the course begins by engaging with
questions about why randomize (or why not randomize) and how randomized
assignment is not the same as random sampling. It then introduces randomization
based statistical inference; an approach to calculating standard errors for
average treatment effect estimates and to calculating $p$-values for tests of
hypotheses about causal effects. Discusses power analysis, and ends by engaging
with some of the trickier issues in experimental design and analysis: what to
do when experimental units drop out of the study or otherwise do not provide
valid outcomes? how might we think about making causal inferences when the
active treatment, the treatment of theoretical interest, cannot be directly
randomized? what to do when we cannot randomly assign treatment directly to
individual units but only to groups of them? Throughout we will be using the R
programming language to demonstrate statistical concepts and also as tools for
designing and analyzing randomized experiments. Most of the examples in the
class will come from field experiments but given the focus on fundamentals the
class will be useful for those working with randomized survey experiments and
randomized lab experiments as well.

I will be open to adding and subtracting topics depending on student interest.

\section{Goals and Expectations}

This class aims to help you get started with the design and analysis of
randomized experiments using randomization as the basis for statistical
inference.

The point of the course is to position you to do the future learning
that is at the core of your work as a researcher. That is,
for most of your life you will not have classes or even textbooks
available to help you learn how to work with data and statistics.
Rather, you will have to learn on your own, with the help of the
internet, friends, and friendly AI. So, this course aims to help you
learn how to learn even as it helps you learn how to reason, decide and
evaluate.

The \textbf{specific goals} of the course are that students:

\begin{itemize}[noitemsep]

    \item Explain in their own words key concepts in statistics like "causal
        inference","statistical inference", "hypothesis testing", "point
        estimation", "p-value", "confidence interval", "random assignment", and
        describe how such concepts fit together in applied research.
    \item Have practiced coding randomization, power calculations, and statistical analysis
        of experimental results
    \item Be familiar with standards and practices for many aspects of reproducible research
    \item Have practiced developing an experimental research design
    \item Have practiced created a pre-analysis plan for an experimental design

\end{itemize}


\subsection{Readings}
\begin{verse}

\fullcite{gerber2012field}

\fullcite{bowersVoorsIchino2021book}

\fullcite{blair2023research}
\end{verse}

I also recommend some chapters from
\href{https://www-jstor-org.eui.idm.oclc.org/stable/j.ctt4cgd52}{\cite{glennerster13}},
which presents material similar to Gerber and Green but less technically.
Reading this volume alongside Gerber and Green might bring together intuition
and statistics for some students.

Other readings will be assigned and distributed electronically.

If you're new to \texttt{R} and/or statistics, I suggest getting a hold of:
\begin{verse}
    \fullcite{fox2016}

    \fullcite{wickhamgrolemund2017}
\end{verse}

\subsection{Expectations}

\textbf{Ask questions when you don't understand things; chances are you're not
alone.}

        I will be open to constructive and concrete suggestions about how to
        teach the class as we go along, and I will value such evaluations at any
        point in the class. I have made changes to this course in the middle of
        the term upon hearing great and useful ideas from students. I am happy
        to do so.

        I assume some previous engagement with high school mathematics,
        probability and statistical computing in the R statistical programming
        language. If you haven't had experience with R but you love learning
        computing languages then you can still get a lot out of this course ---
        you will learn a lot about R as kind of laboratory for learning about
        statistical theory and evaluating and analyzing experimental designs.

        All papers written in this class will use
        \href{http://www.jakebowers.org/PAPERS/11-BOWERS-RCP-363.pdf}{reproducible
        and/or literate programming practices} \parencite{bowers2016future} and
        will include a code appendix.

	All final written work will be turned in as pdf files unless we have
	another specific arrangement.\footnote{For example, if you have some
	reason why pdf files make your life especially difficult, then of course I will work with you find another format.}

        All papers written in this class will assume familiarity with the
        principles of good writing in \textcite{beck:1986}.

        \section{Opportunities for Practice and Learning}

	\textbf{Final project} I'll ask you to apply one or more of the
	approaches we discuss in class to a paper that you are writing. You can
	turn in the whole paper or just a method appendix or perhaps a
	pre-analysis plan using simulated or real data to show how you plan to
	do an analysis (see \cite[Chap 23 on Planning]{rosenbaum2020book} for
	more on analysis plans in general for observational studies).

        \section{General Policies and Perspectives}

        \subsection{Grades are Feedback}

        Humans need feedback to close the intention to action gap. They also need
        feedback to feel good about their progress and to motivate them. In this
        class I will use grades as feedback. All grades map roughly onto A=satisfactory,
        C=unsatisfactory, and F=fail (i.e. you didn't try).

        I'll calculate your grade for the course this way: 40\% attendance  ("A" if you show up,
        ``F'' if not); 30\% weekly work; 30\% final paper.

        You can miss one class and one weekly work without grade penalty.

        Because moments of evaluation are also moments of learning in this
        class, I do not curve. If you all perform at 100\%, then I will give
        you all As.

	You can redo any  evaluation or the final paper to increase your grade
	on that  assignment. If you want to resubmit something already graded,
	you need  to let me know  in advance so that I can make  time to grade
	it again. I will take a long time to grade work that I receive after
	the deadlines: for example, I have taken 6 months or more to grade work
	turned in after final grades are due at the end of the term.

 \end{introstuff}
    %\restoregeometry

 \part*{Schedule}

        \textbf{Note: } This schedule is preliminary and subject to change. If
        you miss a class make sure you contact me or one of your colleagues to
        find out about changes in the lesson plans or assignments.

\titleformat{\subsection}[hang]{\sffamily\bfseries}{\thesubsection}{}{}{}
\titleformat{\subsubsection}[hang]{\small\sffamily\bfseries}{\thesubsubsection}{}{}{}
\titlespacing{\subsection}{0pt}{1.5ex plus .1ex minus .2ex}{0pt}



\textbf{Session One, May 2:}  Causal inference and the potential outcomes framework

What is causality, the potential outcomes framework, and how do experiments help causal identification?

\begin{itemize}
    \item \fullcite{gerber12}, ch. 1 and ch. 2, sections 2.1 and 2.2 only.
    \item \href{https://www-cambridge-org.eui.idm.oclc.org/core/journals/american-political-science-review/article/social-pressure-and-voter-turnout-evidence-from-a-largescale-field-experiment/11E84AF4C0B7FBD1D20C855972C2C3EB}{\fullcite{gerber2008social}}.
    \item \textbf{Recommended:}  \href{https://www-jstor-org.eui.idm.oclc.org/stable/j.ctt4cgd52}{\fullcite{glennerster13}, chs. 1--3.}
    \item \textbf{Recommended:} \href{https://ebookcentral.proquest.com/lib/eui/detail.action?docID=688714}{\fullcite{banerjee2011poor}, ch. 1.}
    \item \textbf{Recommended:}
        \href{https://egap.github.io/theory_and_practice_of_field_experiments/causal-inference.html}{\fullcite{bowersVoorsIchino2021book}},
        Module 3 materials.
\end{itemize}


\textbf{Session Two, May 3:} Random assignment and identification under randomization

Randomization strategies, random sampling versus random assignment, identification under randomization.

\begin{itemize}
    \item \fullcite{gerber12}, ch. 2, remaining sections
    \item \href{https://www-jstor-org.eui.idm.oclc.org/stable/j.ctt21c4v92}{\fullcite{karlan16},} ch. 2 (skim).
    \item \textbf{Recommended:}  \href{https://www-jstor-org.eui.idm.oclc.org/stable/j.ctt4cgd52}{\fullcite{glennerster13},} ch. 4.
    \item \textbf{Recommended:} \href{https://www-cambridge-org.eui.idm.oclc.org/core/journals/world-politics/article/abs/clientelism-and-voting-behavior-evidence-from-a-field-experiment-in-benin/3E386064D15E5E162AEDCEBECB32E8CB}{\fullcite{wantchekon2003clientelism}.}
    \item \textbf{Recommended:}
        \href{https://www-jstor-org.eui.idm.oclc.org/stable/42919295?refreqid=excelsior%3A32c7205b7355e1d97bde057069a83600&seq=1}{\fullcite{collier2014votes}.}
        \item \textbf{Recommended:}
            \href{https://egap.github.io/theory_and_practice_of_field_experiments/randomization.html}{\fullcite{bowersVoorsIchino2021book}}, Module 4 materials.
\end{itemize}

\textbf{Session Three, May 4: } Causal inference

Sampling distributions, causal inference, hypothesis testing.

\begin{itemize}
    \item \fullcite{gerber12}, ch. 3.
    \item \textbf{Recommended:}
        \href{https://egap.github.io/theory_and_practice_of_field_experiments/}{\fullcite{bowersVoorsIchino2021book}},
        Module 5 \& 6 materials.
\end{itemize}

Statistical power.

\begin{itemize}
    \item \fullcite{gerber12}, ch. 3.
    \item \href{https://www-jstor-org.eui.idm.oclc.org/stable/j.ctt21c4v92}{\cite{karlan16},} ch. 5.
    \item \textbf{Recommended:}  \href{https://www-jstor-org.eui.idm.oclc.org/stable/j.ctt4cgd52}{\fullcite{glennerster13},} ch. 6.
    \item \textbf{Recommended:}
        \href{https://egap.github.io/theory_and_practice_of_field_experiments/}{\cite{bowersVoorsIchino2021book}},
        Module 7 materials.
\end{itemize}

\textbf{Session Four: } Student presentations and feedback

\centerline{\textbf{FE II SYLLABUS}}

        \textbf{Note: } This schedule is preliminary and subject to change. If
        you miss a class make sure you contact me or one of your colleagues to
        find out about changes in the lesson plans or assignments.

    \subsection{Useful Reading: About randomized experiments are and their benefits and assumptions}

    *\fullcite{rosenbaum2020book} Sections 2.1--2.4

    *\fullcite[Chap 2--3]{rosenbaum2017}

    *\fullcite[Chap 2--2.4]{rosenbaum2002book} explains and formalizes Fisher's randomization inference.

    *\fullcite[Chap 1--3]{gerber2012field} explains experiments and estimation of average causal effects

    \fullcite[Chap 2]{fisher:1935} explains \emph{the} invention of
    random-assignment based randomization inference in about 15 pages.

    \fullcite{neyman:1923,rubin1990apt} \emph{the} invention of random-sampling
    based randomization inference (estimators of the average causal effect and
    their standard errors).

     \fullcite{bowersleavitt2020} Connects Fisher's test and Neyman's
     estimator for causal inference.

    \AdvanceDate[7]
    \section{\themonth~\the\day---Noncompliance/Instrumental Variables}

    \subsection{Topics} If an intervention has been randomized but the active
    dose has not, what can we do? It turns out that in this case we can
    estimate Complier Average Causal Effects (CACE) (also known as the Local
    Average Treatment Effect (LATE)) and we can also test hypotheses about the
    causal effects of non-randomly taking a dose of a randomized treatment.

\subsection{Useful readings:}

*Chapters 5 and 6 of \fullcite{gerbergreen2012}.

*Chapter 5 and especially Section 5.3, ``Instruments,'' of \fullcite{rosenbaum2020book}

*Chapter 13, ``Instruments'' of \fullcite{rosenbaum2017}

*\fullcite{sovey2011instrumental}

\fullcite{rosenbaum1996}

\fullcite{angristetal1996}

\fullcite{imbensrosenbaum2005} on weak instruments and the problem of 2SLS as an estimator

\fullcite{kangpeckkeele2018}

\fullcite{hansenbowers2009} on a simple design-based approach to what might be
a two-stage instrumental variables multilevel model with binary outcomes.

    \AdvanceDate[7]
    \section{\themonth~\the\day---Simple Stratification and Matching}

    \subsection{Topics} Matching methods as a way to do the
    stratification-based approaches in a way that also allows for assessment of the
    question: ``Did you adjust enough?''

    \subsection{Useful Reading}

    *\fullcite[Chap 5 and 11]{rosenbaum2017}

    *\fullcite[Chap 1,3,7,8,9,10,14]{rosenbaum2020book} (available via SpringerLink from our library)

    *\fullcite[Chap 9.0--9.2]{gelman2007dau} (on causal inference and especially interpolation and extrapolation)

    *\fullcite{hansen:2004} on full matching for adjustment

    *\fullcite{hansen2008cbs} on assessing balance.

    \fullcite{pashley2020blocked} and \fullcite{miratrix2021multisite} on issues in estimating standard errors and overall average effects after matching

    For matching with more than one level such as children nested within schools see \cite{zubizarreta2012using} or \cite{zubizarreta2017optimal} plus \cite{pimentel2018optimal}.


    \AdvanceDate[7]
    \section{\themonth~\the\day---Stratification Using Many Variables}

    \subsection{Topics} Mahalanobis and propensity scores, calipers, penalties,
    exact matching. Maybe: fine balance.

    \subsection{Useful Reading} See readings from last week and this overview from Rosenbaum:

    \fullcite{rosenbaum2020}


    \AdvanceDate[7]
    \section*{\themonth~\the\day---No Class}

    Work on projects. Make an appointment to discuss project with Jake on Calendly.

    \AdvanceDate[7]
    \section{\themonth~\the\day---Stratification When the Intervention is Not Binary}

    \subsection{Topics} Non-bipartite matching.

    \subsection{Useful Reading}

    *\fullcite{rosenbaum2020book}, Chapter 12 on matching without groups

    *\fullcite{rabb2022pnas} This is an example of a published paper using non-bipartite matching.

    \fullcite{zubizarretaetal2013}

    \fullcite{luetal2001}

%    \fullcite{lu2005}
 %   \fullcite{baiocchietal2010}
  %  \fullcite{wongetal2012b}
   % \fullcite{wongetal2020}


    \AdvanceDate[7]
    \section{\themonth~\the\day---Sensitivity Analysis}

    \subsection{Topics} All observational studies leave something out. How big
    must the influence of the unobserved variable be in order to overturn our
    substantive results?

    \subsection{Useful Reading}

    *\fullcite{rosenbaum2017}[Chapter 9]

    *\fullcite{rosenbaum2020book}[Chapter 3]

    *\fullcite{cinellihazlett2020}

    \fullcite{hosmanetal2010}

    \fullcite{imbens2003}

   % \fullcite{oster2019}

    \fullcite{rosenbaum2018}

    \fullcite{rosenbaum2009a}

    \fullcite{hansenrosenbaumsmall2014}

     \fullcite{hsusmall2013}

    \fullcite{chaudoin2018we}

    \subsection{Useful reading about sensitivity analysis for weak null hypotheses}


 \fullcite{fogarty2023}

 \fullcite{fogarty2020b}

 \fullcite{fogartyetal2017}


%    \fullcite{rabb2022pnas}

    \AdvanceDate[7]
    \section*{\themonth~\the\day---No Class}

    Work on projects. Make an appointment to discuss project with Jake on Calendly.

    \AdvanceDate[7]
    \section{\themonth~\the\day--- Stratification With Longitudinal Data}

    \subsection{Topics} How to make clear comparisons in pre-post designs, risk-set matching, time-series
    cross-sectional data, synthetic control methods.

    \subsection{Useful reading}

    *\fullcite{rosenbaum2020book}, Chapter 13

    *\fullcite{rosenbaum2017}, Chapter 11

    TBA

    \AdvanceDate[7]
    \section{\themonth~\the\day--- No Class}

    \AdvanceDate[7]
    \section{\themonth~\the\day--- TBD}

    \subsection{Topics} To be decided by the class. Topics could include:  Difference in Differences OR Discontinuity Designs


\bigskip

\textbf{Plan to meet with Jake at least once between Oct 25 and when the final paper is
due.}

    \SetDate[8/12/2023]
    \section*{\themonth~\the\day---Final Assignment Due}
    If you would like to turn in your final assignment after this date, please let me
    know at least a week in advance.


    \printbibliography[title=References]

    \end{document}

